%!TEX program = xelatex
%!BIB program = bibtex

\documentclass[cn,black,12pt,normal]{elegantnote}
\usepackage{float}
\usepackage{hyperref}

\newcommand{\upcite}[1]{\textsuperscript{\textsuperscript{\cite{#1}}}}

\title{PRML读书笔记手稿整理}
\author{PRML讨论班}
\institute{Tongji University}
\version{0.01}
\date{\zhtoday}

\begin{document}

%\maketitle
%\tableofcontents
%\newpage

\section{变分法}

$F[y]$泛函。

% 注:泛函(functional)通常是指\textbf{定义域为函数集},而\textbf{值域为实数或者复数}的\textbf{映射},换句话说,它是从函数组成的一个向量空间到标量域的映射,它的输入为函数,而输出为标量。

\begin{equation*}
    F(y(x)+\epsilon \eta(x) ) = F(y(x)) + \eta \int \frac{\partial F}{\partial y(x)}\eta(x) dx + O(\epsilon^2)
\end{equation*}

\begin{equation*}
    \int\limits_{x}^{}  \frac{\partial F}{\partial y(x)}\eta(x) dx = 0
\end{equation*}

\section{高斯分布的矩}

\paragraph{一阶原点矩} 

\begin{align*}
E(x) & = \frac{1}{(2\pi )^{\frac{D }{2} }} \frac{1}{\left | \Sigma  \right |^{\frac{1}{2} }  }\int 
\exp\left \{ -\frac{1}{2} (x-\mu )^T\Sigma^{-1}(x-\mu) \right \}x dx  
\end{align*}
做换元$z = x - \mu$ 后,有:
\begin{align*}
E(x) & = \frac{1}{(2\pi )^{\frac{D }{2} }} \frac{1}{\left | \Sigma  \right |^{\frac{1}{2} }  }\int 
\exp\left \{ -\frac{1}{2} z^T\Sigma^{-1}z \right \}(z+\mu) dz  
\end{align*}
而由对称性:
\begin{align*}
\int \exp\left \{ -\frac{1}{2} z^T\Sigma^{-1}z \right \}(z) dz = 0
\end{align*}
故有:
\begin{align*}
E(x) & = \frac{1}{(2\pi )^{\frac{D }{2} }} \frac{1}{\left | \Sigma  \right |^{\frac{1}{2} }  }\int 
\exp\left \{ -\frac{1}{2} z^T\Sigma^{-1}z \right \}(z+\mu) dz = \mu
\end{align*}

\paragraph{二阶原点矩}
类似地,做换元$z = x - \mu$ 后,有:
\begin{align*}
E(x) & = \frac{1}{(2\pi )^{\frac{D }{2} }} \frac{1}{\left | \Sigma  \right |^{\frac{1}{2} }  }\int 
\exp\left \{ -\frac{1}{2} z^T\Sigma^{-1}z \right \}(z+\mu)(z+\mu)^T dz  
\end{align*}
将$(z+\mu)(z+\mu)$展开,考虑对称性,有:
\begin{align*}
E(x) & = \mu\mu^T \\ & + \frac{1}{(2\pi )^{\frac{D }{2} }} \frac{1}{\left | \Sigma  \right |^{\frac{1}{2} }  }\int 
\exp\left \{ -\frac{1}{2} z^T\Sigma^{-1}z \right \} zz^T dz  
\end{align*}
考虑$I = \frac{1}{(2\pi )^{\frac{D }{2} }} \frac{1}{\left | \Sigma  \right |^{\frac{1}{2} }  }\int 
\exp\left \{ -\frac{1}{2} z^T\Sigma^{-1}z \right \} zz^T dz$
\end{document}
