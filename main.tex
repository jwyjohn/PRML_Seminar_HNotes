%!TEX program = xelatex
%!BIB program = bibtex

\documentclass[cn,black,12pt,normal]{elegantnote}
\usepackage{float}
\usepackage{hyperref}
\usepackage{amssymb}

\newcommand{\upcite}[1]{\textsuperscript{\textsuperscript{\cite{#1}}}}

\title{PRML读书笔记手稿整理}
\author{PRML讨论班}
\institute{Tongji University}
\version{0.01}
\date{\zhtoday}

\begin{document}

%\maketitle
%\tableofcontents
%\newpage

\section*{写在前面}

需熟悉常见的矩阵微积分表示方法及其具体含义。

\section{变分法}

$F[y]$泛函。

% 注:泛函(functional)通常是指\textbf{定义域为函数集},而\textbf{值域为实数或者复数}的\textbf{映射},换句话说,它是从函数组成的一个向量空间到标量域的映射,它的输入为函数,而输出为标量。

\begin{equation*}
    F(y(x)+\epsilon \eta(x) ) = F(y(x)) + \eta \int \frac{\partial F}{\partial y(x)}\eta(x) dx + O(\epsilon^2)
\end{equation*}

\begin{equation*}
    \int\limits_{x}^{}  \frac{\partial F}{\partial y(x)}\eta(x) dx = 0
\end{equation*}

\section{高斯分布的矩}

\paragraph{一阶原点矩}

\begin{align*}
    E(x) & = \frac{1}{(2\pi )^{\frac{D }{2} }} \frac{1}{\left | \Sigma  \right |^{\frac{1}{2} }  }\int
    \exp\left \{ -\frac{1}{2} (x-\mu )^T\Sigma^{-1}(x-\mu) \right \}x dx
\end{align*}
做换元$z = x - \mu$ 后,有:
\begin{align*}
    E(x) & = \frac{1}{(2\pi )^{\frac{D }{2} }} \frac{1}{\left | \Sigma  \right |^{\frac{1}{2} }  }\int
    \exp\left \{ -\frac{1}{2} z^T\Sigma^{-1}z \right \}(z+\mu) dz
\end{align*}
而由对称性:
\begin{align*}
    \int \exp\left \{ -\frac{1}{2} z^T\Sigma^{-1}z \right \}(z) dz = 0
\end{align*}
故有:
\begin{align*}
    E(x) & = \frac{1}{(2\pi )^{\frac{D }{2} }} \frac{1}{\left | \Sigma  \right |^{\frac{1}{2} }  }\int
    \exp\left \{ -\frac{1}{2} z^T\Sigma^{-1}z \right \}(z+\mu) dz = \mu
\end{align*}

\paragraph{二阶原点矩}
类似地,做换元$z = x - \mu$ 后,有:
\begin{align*}
    E(x) & = \frac{1}{(2\pi )^{\frac{D }{2} }} \frac{1}{\left | \Sigma  \right |^{\frac{1}{2} }  }\int
    \exp\left \{ -\frac{1}{2} z^T\Sigma^{-1}z \right \}(z+\mu)(z+\mu)^T dz
\end{align*}
将$(z+\mu)(z+\mu)^T$展开,考虑对称性,有:
\begin{align*}
    E(x) & = \mu\mu^T \\ & + \frac{1}{(2\pi )^{\frac{D }{2} }} \frac{1}{\left | \Sigma  \right |^{\frac{1}{2} }  }\int
    \exp\left \{ -\frac{1}{2} z^T\Sigma^{-1}z \right \} zz^T dz
\end{align*}
考虑$I = \frac{1}{(2\pi )^{\frac{D }{2} }} \frac{1}{\left | \Sigma  \right |^{\frac{1}{2} }  }\int
    \exp\left \{ -\frac{1}{2} z^T\Sigma^{-1}z \right \} zz^T dz$
又有:$z = U^{-1}y = \Sigma y_ju_j$,即$y=Uz$
\begin{align*}
    I & = \frac{1}{(2\pi )^{\frac{D }{2} }} \frac{1}{\left | \Sigma  \right |^{\frac{1}{2} }  }\sum_{i,j}u_iu_j^T \int 1 \cdot
    \exp\left \{ -\sum_{k=1}^{D} \frac{y_k^2}{2\lambda_k} \right \} y_i y_j d\vec{y}
\end{align*}
再次利用对称性,当且仅当$i=j$时,有:
\begin{align*}
    \sum_{i,j}u_iu_j^T \int 1 \cdot
    \exp\left \{ -\sum_{k=1}^{D} \frac{y_k^2}{2\lambda_k} \right \} y_i y_j d\vec{y} & \neq 0
\end{align*}
所以:
\begin{align*}
    I & = \sum_{i=1}{D}u_iu_i^T \frac{1}{(2\pi )^{\frac{D }{2} }} \frac{1}{\left | \Sigma  \right |^{\frac{1}{2} }  } \int 1 \cdot
    \exp\left \{ -\sum_{k=1}^{D} \frac{y_k^2}{2\lambda_k} \right \} y_i y_j d\vec{y}                                               \\ & = \sum_{i=1}{D}u_iu_i^T \lambda_i = \Sigma
\end{align*}

\paragraph{二阶中心矩}
\begin{align*}
    \text{Var}(x) & = E((X-E(X))(X-E(X))^T)
    \\ & = E(xx^T) - E(x)(E(x))^T = \Sigma
\end{align*}

\section{条件高斯分布}

\paragraph{划分向量}
\begin{align*}
    \begin{bmatrix}
        a_1    \\
        \vdots \\
        a_m    \\
        b_1    \\
        \vdots \\
        b_n
    \end{bmatrix} = \begin{bmatrix}
                          a \\
                          b
                      \end{bmatrix},\; \mu = \begin{bmatrix}
                                                  \mu_a \\
                                                  \mu_b
                                              \end{bmatrix},\; \Sigma = \begin{bmatrix}
                                                                             \Sigma_{aa} & \Sigma_{ab} \\
                                                                             \Sigma_{ba} & \Sigma_{bb}
                                                                         \end{bmatrix}
\end{align*}

\paragraph{精度矩阵}
\begin{align*}
    \Lambda & = \Sigma^{-1} = \begin{bmatrix}
                                  \Lambda_{aa} & \Lambda_{ab} \\
                                  \Lambda_{ba} & \Lambda_{bb}
                              \end{bmatrix}
\end{align*}
值得注意的是:$\Lambda_{ij}\neq\Sigma_{ij}$(分块的性质)

\section{Remark. Schur补}

对于矩阵$\begin{bmatrix}
            A & B \\
            C & D
        \end{bmatrix}$
,若$D \in \text{GLn}(\mathbb{F} )$,则:

称$A-BD^{-1}C$为$D$关于$M$的Schur补;

称$D-CA^{-1}B$为$A$关于$M$的Schur补。
\paragraph{求解Schur补的过程如下:}
\begin{align*}
    \begin{bmatrix}
        I & 0 \\
        -CA^{-1} & I
    \end{bmatrix}\begin{bmatrix}
        A & B \\
        C & D
    \end{bmatrix} & = \begin{bmatrix}
        A & B \\
        0 & \Delta_A
    \end{bmatrix} \\
    \begin{bmatrix}
        A & B \\
        C & D
    \end{bmatrix}\begin{bmatrix}
        I & -A^{-1}B \\
        0 & I
    \end{bmatrix} & = \begin{bmatrix}
        I & 0 \\
        C & \Delta_A
    \end{bmatrix} \\
\end{align*}
故有:
\begin{align*}
    \begin{bmatrix}
        I & 0 \\
        -CA^{-1} & I
    \end{bmatrix}\begin{bmatrix}
        A & B \\
        C & D
    \end{bmatrix}\begin{bmatrix}
        I & -A^{-1}B \\
        0 & I
    \end{bmatrix} & = \begin{bmatrix}
        A & 0 \\
        0 & \Delta_A
    \end{bmatrix} \\
    \begin{bmatrix}
        I & 0 \\
        -CA^{-1} & I
    \end{bmatrix}\begin{bmatrix}
        A & 0 \\
        0 & \Delta_A
    \end{bmatrix} \begin{bmatrix}
        I & -A^{-1}B \\
        0 & I
    \end{bmatrix} & = \begin{bmatrix}
        A & B \\
        C & D
    \end{bmatrix} 
\end{align*}
所以:
\begin{align*}
    \begin{bmatrix}
        A & B \\
        C & D
    \end{bmatrix}^{-1} & = 
    \begin{bmatrix}
        I & 0 \\
        -CA^{-1} & I
    \end{bmatrix}\begin{bmatrix}
        A^{-1} & 0 \\
        0 & \Delta_A^{-1}
    \end{bmatrix} \begin{bmatrix}
        I & -A^{-1}B \\
        0 & I
    \end{bmatrix}
\end{align*}
\textit{Rmk. 从上面的过程可以推出有}$\Delta_A = D - CA^{-1}B \in \text{GLn}(\mathbb{F} )$(做法类似高斯消元)
\\进而:
\begin{align*}
    \begin{bmatrix}
        A & B \\
        C & D
    \end{bmatrix}^{-1} & = 
    \begin{bmatrix}
        A^{-1}+A^{-1}B\Delta_ACA^{-1} & A^{-1}B\Delta_A \\
        -\Delta_ACA^{-1} & \Delta_A
    \end{bmatrix}
\end{align*}







\end{document}